
% Journal Article
% LaTeX Template
% Version 1.4 (15/5/16)
%
% This template has been downloaded from:
% http://www.LaTeXTemplates.com
%
% Original author:
% Frits Wenneker (http://www.howtotex.com) with extensive modifications by
% Vel (vel@LaTeXTemplates.com)
%
% License:
% CC BY-NC-SA 3.0 (http://creativecommons.org/licenses/by-nc-sa/3.0/)
%
%%%%%%%%%%%%%%%%%%%%%%%%%%%%%%%%%%%%%%%%%

%----------------------------------------------------------------------------------------
%	PACKAGES AND OTHER DOCUMENT CONFIGURATIONS
%----------------------------------------------------------------------------------------

\documentclass[twoside,twocolumn]{article}

\usepackage{blindtext} % Package to generate dummy text throughout this template 

\usepackage[sc]{mathpazo} % Use the Palatino font
\usepackage[T1]{fontenc} % Use 8-bit encoding that has 256 glyphs
\linespread{1.05} % Line spacing - Palatino needs more space between lines
\usepackage{microtype} % Slightly tweak font spacing for aesthetics

\usepackage[english]{babel} % Language hyphenation and typographical rules

\usepackage[hmarginratio=1:1,top=32mm,columnsep=20pt]{geometry} % Document margins
\usepackage[hang, small,labelfont=bf,up,textfont=it,up]{caption} % Custom captions under/above floats in tables or figures
\usepackage{booktabs} % Horizontal rules in tables

\usepackage{lettrine} % The lettrine is the first enlarged letter at the beginning of the text

\usepackage{enumitem} % Customized lists
\setlist[itemize]{noitemsep} % Make itemize lists more compact

\usepackage{abstract} % Allows abstract customization
\renewcommand{\abstractnamefont}{\normalfont\bfseries} % Set the "Abstract" text to bold
\renewcommand{\abstracttextfont}{\normalfont\small\itshape} % Set the abstract itself to small italic text

\usepackage{titlesec} % Allows customization of titles
\renewcommand\thesection{\Roman{section}} % Roman numerals for the sections
\renewcommand\thesubsection{\roman{subsection}} % roman numerals for subsections
\titleformat{\section}[block]{\large\scshape\centering}{\thesection.}{1em}{} % Change the look of the section titles
\titleformat{\subsection}[block]{\large}{\thesubsection.}{1em}{} % Change the look of the section titles

\usepackage{fancyhdr} % Headers and footers
\pagestyle{fancy} % All pages have headers and footers
\fancyhead{} % Blank out the default header
\fancyfoot{} % Blank out the default footer
\fancyhead[C]{Image Colorization $\bullet$ Sep 2019} % Custom header text
\fancyfoot[RO,LE]{\thepage} % Custom footer text

\usepackage{titling} % Customizing the title section

\usepackage{hyperref} % For hyperlinks in the PDF

\usepackage{longtable}

\usepackage{multicol}

%----------------------------------------------------------------------------------------
%	TITLE SECTION
%----------------------------------------------------------------------------------------

\setlength{\droptitle}{-4\baselineskip} % Move the title up

\pretitle{\begin{center}\Huge\bfseries} % Article title formatting
\posttitle{\end{center}} % Article title closing formatting
\title{Image Colorization} % Article title
\author{%
\textsc{Tommaso Loss, Matteo Marcuzzo} %\thanks{A thank you or further information}
 \\[1ex] % Your name
\normalsize University of Padua \\ % Your institution
%\normalsize \href{mailto:john@smith.com}{john@smith.com} % Your email address
%\and % Uncomment if 2 authors are required, duplicate these 4 lines if more
%\textsc{Jane Smith}\thanks{Corresponding author} \\[1ex] % Second author's name
%\normalsize University of Utah \\ % Second author's institution
%\normalsize \href{mailto:jane@smith.com}{jane@smith.com} % Second author's email address
}
\date{\today} % Leave empty to omit a date
\renewcommand{\maketitlehookd}{%
\begin{abstract}
\noindent \blindtext % Dummy abstract text - replace \blindtext with your abstract text
\end{abstract}
}

%----------------------------------------------------------------------------------------

\begin{document}

% Print the title
\maketitle

%----------------------------------------------------------------------------------------
%	ARTICLE CONTENTS
%----------------------------------------------------------------------------------------

\section{Introduction}

\lettrine[nindent=0em,lines=3]{L} orem ipsum dolor sit amet, consectetur adipiscing elit.
\blindtext % Dummy text

\blindtext % Dummy text

%------------------------------------------------

\section{Implementation}

\subsection{Color space}

\subsection{Loss function}

\begin{equation}
\label{eq:emc}
e = mc^2
\end{equation}


\subsection{Class-rebalancing}

\subsection{Network architecture}

\begin{table*}[t]
\centering
\captionof{table}{Network structure}

\begin{tabular}{|c|c|c|c|c|c|}
	\hline
	\hline
	Layer & Filters & Spac. Res. & Kernel size & Stride & Dilation\\ 
	\hline
	\hline
	\multicolumn{6}{|c|}{input data (with size e.g. 224 * 224)}\\ \hline
	conv1\_1 & 64 & 224 & 3x3 & (1,1) & 1 \\ \hline
	conv1\_2 & 64 & 112 & 3x3 & (2,2) & 1 \\ \hline
	\multicolumn{6}{|c|}{Batch normalization}\\ \hline
	conv2\_1 & 128 & 112 & 3x3 & (1,1) & 1 \\ \hline
	conv2\_2 & 128 & 56 & 3x3 & (2,2) & 1 \\ \hline
	\multicolumn{6}{|c|}{Batch Normalization}\\ \hline
	conv3\_1 & 256 & 56 & 3x3 & (1,1) & 1 \\ \hline
	conv3\_2 & 256 & 56 & 3x3 & (1,1) & 1 \\ \hline
	conv3\_3 & 256 & 28 & 3x3 & (2,2) & 1 \\ \hline
	\multicolumn{6}{|c|}{Batch normalization}\\ \hline
	conv4\_1 & 512 & 28 & 3x3 & (1,1) & 1 \\ \hline
	conv4\_2 & 512 & 28 & 3x3 & (1,1) & 1 \\ \hline
	conv4\_3 & 512 & 28 & 3x3 & (1,1) & 1 \\ \hline
	\multicolumn{6}{|c|}{Batch normalization}\\ \hline
	conv5\_1 & 512 & 28 & 3x3 & (1,1) & 2 \\ \hline
	conv5\_2 & 512 & 28 & 3x3 & (1,1) & 2 \\ \hline
	conv5\_3 & 512 & 28 & 3x3 & (1,1) & 2 \\ \hline
	\multicolumn{6}{|c|}{Batch normalization}\\ \hline
	conv6\_1 & 512 & 28 & 3x3 & (1,1) & 2 \\ \hline
	conv6\_2 & 512 & 28 & 3x3 & (1,1) & 2 \\ \hline
	conv6\_3 & 512 & 28 & 3x3 & (1,1) & 2 \\ \hline
	\multicolumn{6}{|c|}{Batch normalization}\\ \hline
	conv7\_1 & 256 & 28 & 3x3 & (1,1) & 1 \\ \hline
	conv7\_2 & 256 & 28 & 3x3 & (1,1) & 1 \\ \hline
	conv7\_3 & 256 & 28 & 3x3 & (1,1) & 1 \\ \hline
	\multicolumn{6}{|c|}{Batch normalization}\\ \hline
	\multicolumn{6}{|c|}{2D Upsampling with size 2*2}\\ \hline
	conv8\_1 & 128 & 56 & 3x3 & (0.5,0.5)\footnote{The .5 stride is simulated through the 2d upsampling. Earlier implementations made instead use of a deconvolutional layer, but we found this other approach to work better in practice.
	}. & 1 \\ \hline
	conv8\_2 & 128 & 56 & 3x3 & (1,1) & 1 \\ \hline
	conv8\_3 & 128 & 56 & 3x3 & (1,1) & 1 \\ \hline
	\multicolumn{6}{|c|}{Softmax Layer}\\ \hline
	
\end{tabular}
\end{table*}

\subsection{Parameters}

\subsection{Prediction evaluation}

Text requiring further explanation\footnote{Example footnote}.

%------------------------------------------------


\section{Experimentation}

\subsection{Subsection One}

A statement requiring citation \cite{Figueredo:2009dg}.
\blindtext % Dummy text


%------------------------------------------------

\section{Considerations}

\subsection{Recent developments}

%------------------------------------------------

%------------------------------------------------

\section{Conclusions}

%------------------------------------------------

\blindtext % Dummy text

%----------------------------------------------------------------------------------------
%	REFERENCE LIST
%----------------------------------------------------------------------------------------

\begin{thebibliography}{99} % Bibliography - this is intentionally simple in this template

\bibitem[Figueredo and Wolf, 2009]{Figueredo:2009dg}
Figueredo, A.~J. and Wolf, P. S.~A. (2009).
\newblock Assortative pairing and life history strategy - a cross-cultural
  study.
\newblock {\em Human Nature}, 20:317--330.
 
\end{thebibliography}

%----------------------------------------------------------------------------------------

\end{document}
